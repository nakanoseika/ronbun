% !TEX encoding = UTF-8 Unicode

\documentclass[12pt,a4j,titlepage]{ltjsarticle}
\usepackage{semi}
\usepackage{here}
\usepackage{listings}

% \title{}
% \author{}
% \date{}

\begin{document}

\begin{titlepage}
  \begin{center}
  
    \vspace*{20truept}
    
    {\LARGE 2024年度 卒業論文} 
    
    \vspace*{75truept}
    
    {\Huge  タイトル} %論文タイトル

    \vspace{10truept}

    {\Huge } %論文タイトル 長い場合 改行1

    \vspace{10truept}

    {\Huge } %論文タイトル 改行2

    \vspace{85truept}
    
    {\LARGE 指導教員 須田 宇宙 准教授}
    
    \vspace{60truept}
    
    {\LARGE 千葉工業大学 情報ネットワーク学科}
    
    \vspace{15truept}
    
    {\LARGE 須田研究室}
    
    \vspace{70truept}
    
    {\LARGE 2132100 氏名 中野 星花 } % 氏名は消さない 学生番号 氏名 名前

    \vspace{70truept}
    
  \end{center}
  \begin{flushright}

    {\LARGE 提出日 2024年1月17日}
  
  \end{flushright}
\end{titlepage}

\setcounter{tocdepth}{3}
% 目次の出力
\tableofcontents
% 表目次
\listoftables
% 図目次
\listoffigures
\clearpage

\section{緒言}\label{緒言}
%背景


%問題点


%目的


\clearpage

\section{学習について}\label{学習について}


\section{体系化について}
\subsection{概要}

\subsection{対話による体系化の効果}

\subsection{対話相手の違いと体系化への影響}

\section{OSI参照モデル}
OSI参照モデルとは,コンピュータが通信するために利用するネットワークの機能を7つの階層(レイヤー)に分類して整理したモデルである.
それぞれ独立した役割を持つ7つの階層が,互いに連携して通信を実現している.
また,データをやり取りする際,機種や通信方式といった様々な違いを考慮しなければならない.
データの送信側と受信側のコンピュータのプログラムがあらかじめ決められたマニュアル(プロトコル)に沿って通信する.
OSI参照モデルでは,階層ごとに通信プロトコルが定義されている.

\subsection{階層化}
送信側の各階層でデータが処理される際,上位の階層から受け取ったデータに,その階層独自の情報(ヘッダ)を付加して下位の階層に引き渡す.
ヘッダには,その階層での通信制御やエラー検出,送信先情報など,通信を適切に管理するための情報が含まれている.
また,データリンク層ではデータの末尾にトレーラを付加する場合がある.
トレーラには,データが伝送中に損傷していないかを確認するためのエラー検出用の情報が含まれる.
一方で,受信側ではこれが逆順に処理され,各階層で対応するヘッダやトレーラが解析されることで,元のデータがアプリケーション層に復元される.

\subsection{アプリケーション層}
ユーザーが操作するアプリケーションの中で通信に関係する具体的な機能についての仕様や手順を定めている.
ファイル転送や電子メール,遠隔ログインなどを実現するためのプロトコルがある.

\subsection{プレゼンテーション層}
データの表現形式を定義する.
具体的に,アプリケーションが扱う情報を通信に適したデータ形式にしたり,また下位層から来たデータを上位層が処理できるデータ形式にするなどがある.

\subsection{セッション層}
通信プログラムの確立や切断,転送するデータの切れ目の設定など一連の手順(セッション)を定義する.

\subsection{トランスポート層}
宛先のアプリケーションにデータを確実に届ける際,通信の品質をコントロールする層である.
信頼性重視やリアルタイム性重視など,用途に応じてプロトコルを使い分ける.
通信を行う両端のノード(機器)だけで処理され,途中のルーターでは処理されない.

\subsection{ネットワーク層}
宛先までデータを届ける役割を持つ.
宛先が複数のネットワークアドレスでつながった先にある場合には,アドレス体系決めや,通信経路選択(ルーティング)の役割を持つ.

\subsection{データリンク層}
物理層で直接接続されたノード間での通信を可能にする.
0と1の数字の列を意味のあるかたまり(フレーム)に分けて,相手に伝える.

\subsection{物理層}
ビットの列(0と1の数字の列)を物理的な信号に変換し,実際の通信媒体を通じて送信する.





\clearpage



\section{謝辞}
本研究の遂行及び本論文の作成にあたり,須田研究室の仲間に多くの手助けを頂きました,深く感謝の意を表します.そして,本論文の作成にあたり多大なる御指導及び御助言を頂きました,須田宇宙准教授に深く感謝の意を表します.

\clearpage

%参考文献
\begin{thebibliography}{99}
\bibitem{sasaki2012} 佐々木 尚之: ``不確実な時代の結婚-JGSSライフコース調査による潜在的稼得力の影響の検証'', 「家族社会学研究」,第24号,pp152-164(2012)
\bibitem{naikakufu2019} 内閣府: ``少子化対策の現状'', \url{https://www8.cao.go.jp/shoushi/shoushika/whitepaper/measures/w-2016/28webhonpen/html/b1_s1-1-3.html}, 2023/1/16参照
\bibitem{doukou}国立社会保障・人口問題研究所:``第15回出生動向基本調査'',
\url{https://www.ipss.go.jp/ps-doukou/j/doukou15/NFS15_report3.pdf},2023/1/14参照
\bibitem{prtimes}PRTIMES:``マッチングアプリは怖い?危ない目に合った?初めて会うまでの期間は!?徹底調査'',
\url{https://prtimes.jp/main/html/rd/p/000000016.000059676.html},2023/1/14参照
\bibitem{huuhuutyousa}「いい夫婦の日」に関するアンケート調査:``明治安田生命 NEWS RELEASE'',
\url{https://www.meijiyasuda.co.jp/profile/news/release/2022/pdf/20221116_01.pdf},2023/1/16参照
\bibitem{wecsocketsetumi}WebSocketとは?WebSocketについて詳しく解説します:``Web会議の基礎知識'',
\url{https://www.freshvoice.net/knowledge/word/6323/},2023/1/10参照
\end{thebibliography}

\end{document}
