% !TEX encoding = UTF-8 Unicode

\documentclass[12pt,a4j,titlepage]{ltjsarticle}
\usepackage{semi}
\usepackage{here}
\usepackage{listings}

% \title{}
% \author{}
% \date{}

\begin{document}

\begin{titlepage}
  \begin{center}
  
    \vspace*{20truept}
    
    {\LARGE 2024年度 卒業論文} 
    
    \vspace*{75truept}
    
    {\Huge  タイトル} %論文タイトル

    \vspace{10truept}

    {\Huge } %論文タイトル 長い場合 改行1

    \vspace{10truept}

    {\Huge } %論文タイトル 改行2

    \vspace{85truept}
    
    {\LARGE 指導教員 須田 宇宙 准教授}
    
    \vspace{60truept}
    
    {\LARGE 千葉工業大学 情報ネットワーク学科}
    
    \vspace{15truept}
    
    {\LARGE 須田研究室}
    
    \vspace{70truept}
    
    {\LARGE 2132100 氏名 中野 星花 } % 氏名は消さない 学生番号 氏名 名前

    \vspace{70truept}
    
  \end{center}
  \begin{flushright}

    {\LARGE 提出日 2024年1月17日}
  
  \end{flushright}
\end{titlepage}

\setcounter{tocdepth}{3}
% 目次の出力
\tableofcontents
% 表目次
\listoftables
% 図目次
\listoffigures
\clearpage

\section{緒言}\label{緒言}
%背景


%問題点


%目的


\clearpage

\section{協同学習について}
協同学習とは「主体的で自律的な学びの構え,確かで幅広い知的習得,仲間と共に課題解決に向かうことのできる対人技能,さらには,他者を尊重する民主的な態度,といった『学力』を効率的に身につけていくための『基本的な考え方』」を指す.

学習者が主体的で自律的な学力をつけることは,一対多の講義形式のような,学習者が受け身で学ぶ授業では困難である.
なぜなら受け身の学習では,知識や技能が,教師から学習者へと受け渡しされ,授業の中で学習者にどのような学びが起こっているのかは重要ではなく,テストなどの結果に関心が置かれるからである.
従って,学習者自身が授業で教わった知識を主体的に深めるために,他者と協働しながら問題解決に挑む学習方略として協同学習が必要である.
協同学習では,学習者同士が意見を交換し,異なる視点や考え方に触れることで,多様な理解が促進される.
これにより,個人の思考が広がり,より豊かな学びが実現する.
また,仲間と共に学ぶ過程で,コミュニケーション能力や対人関係のスキルが育まれ,他者を尊重する態度が培われる.

このように,協同学習は,学力向上だけでなく,社会で求められる幅広い能力を効果的に育成するための重要な考え方である.

\section{体系化について}
\subsection{概要}
体系化とは,物事や情報を一定の決まりに基づいて整理させることである.それぞれの要素が互いに関連し,全体として一貫性を持たせることで,情報の理解を容易にする.

\subsection{学習における体系化の重要性}
学習において体系化は,学習内容を整理し,論理的な順序や構造に基づいて組み立てることで,知識の定着を助け,応用力や問題解決能力を向上させるために有効である.

%\subsection{対話相手の違いと体系化への影響}
%対話相手の違いとは,主に相手との親密度,相手の知識量,相手の知識の体系化の程度が挙げられる.

%親しい相手との対話では,親しくない相手と比較すると,自由な発言と試行錯誤が促される点がある.カジュアルに意見を交わしやすいことによる違いが生じる.

%相手が自分より知識量が多い場合,質問を通じて理解を深める機会が多いが,議論が一方的になり,主体的な思考を妨げる可能性がある.一方,自分より相手の知識量が少ない場合,相手に教える立場になりやすいため,分かりやすく伝えるために,単純化や具体例を交えることで理解が深まる機会が多い.しかし,この場合も議論が一方的になりやすいという問題がある.


\section{演習・実習について}
大学のカリキュラムに,演習・実習を体系的に配置することで,自主的な姿勢で課題に取り組むように指導を実施している.

また,思考・判断のプロセスを説明するためのプレゼンテーション能力やコミュニケーション能力,グループでの共同作業を適確に実行し,協力関係をつくり上げていく能力を育成することが目的である.

\subsection{ネットワーク管理実習}
\subsection{概要}
ネットワーク管理実習は,第6セメスターに開校された.
この授業の目的は,インターネットの普及に伴い,各種組織から一般家庭に至るまで幅広く利用されているTCP/IPを基盤としたローカルエリアネットワーク(LAN)の管理に必要なスキルを習得することにある.
しかし,LANの設計や構築を担う技術者は必ずしも十分に確保されているとは言えず,こうした状況に対応するためには実践的な学習の場が求められている.
このような背景を踏まえ,本実習では,LANの構成単位であるサブドメインの設計・構築を行い,LANの構築に応用することで,LAN管理に必要な実践的スキルの習得を目指す.

\subsubsection{ルーティング}
インターネットでは,IPアドレッシングに基づき割り当てられたネットワークが互いに接続されている.
これらのネットワーク間でパケットを転送することをルーティングといい,この中継を行う装置をルータという.
ここで,ルータは接続するネットワークの数だけネットワークインタフェースを持つ.

また,離れたホストへ通信する際に多くのルータを経由する.
この順路のことをルートという.
そのため,ルータはパケットを中継する際にどのルータへパケットを渡せば目的のホストにパケットが届けられるかを知る必要がある.
この情報を経路情報という.
経路情報には目的のネットワークの存在する方向(インタフェース)とその優先度が書かれたルーティングテーブルを利用する.

\subsubsection{名前解決}
名前解決とは,IPアドレスとドメイン名であるFQDNを相互に変換することを指す.
この時,FQDNからIPアドレスを調べることを「正引き名前解決」と呼び,IPアドレスからFQDNを調べることを「逆引き名前解決」という.
また,この仕組みを提供するサーバをDNSサーバという.

\subsubsection{SMTPサーバ}
SMTPサーバとは,メール送信の際に必要となるサーバーである.
メールを送信の命令を受け取ったSMTPサーバーは,送信メールを,送信先メールアドレスを管理するSMTPサーバーまで送る.




\section{OSI参照モデル}
\subsection{概要}
OSI参照モデルとは,コンピュータが通信するために利用するネットワークの機能を7つの階層(レイヤー)に分類して,機能を分割することで,複雑なネットワークプロトコルを単純化したモデルである.
最初のネットワークの実装では,各メーカーや組織独自の通信プロトコルを用いていた.
しかし,このモデルを用いるようになったことで,ネットワークの設計が体系化され,標準化が進展した.
また,データをやり取りする際,機種や通信方式といった様々な違いを考慮しなければならない.
データの送信側と受信側のコンピュータのプログラムがあらかじめ決められたマニュアル(プロトコル)に沿って通信し,OSI参照モデルに準拠するよう各種プロトコルが作られている.

\subsubsection{階層化}
アプリケーション層は,ユーザーが操作するアプリケーションの中で通信に関係する具体的な機能についての仕様や手順を定めて過程で
ファイル転送や電子メール,遠隔ログインなどを実現するためのプロトコルがある.

プレゼンテーション層では,データの表現形式を定義する.
具体的に,アプリケーションが扱う情報を通信に適したデータ形式にしたり,また下位層から来たデータを上位層が処理できるデータ形式にするなどがある.

セッション層は,通信プログラムの確立や切断,転送するデータの切れ目の設定など一連の手順(セッション)を定義する.

トランスポート層は,宛先のアプリケーションにデータを確実に届ける際,通信の品質をコントロールする層である.
信頼性重視やリアルタイム性重視など,用途に応じてプロトコルを使い分ける.
通信を行う両端のノード(機器)だけで処理され,途中のルーターでは処理されない.

ネットワーク層は,宛先までデータを届ける役割を持つ.
宛先が複数のネットワークアドレスでつながった先にある場合には,アドレス体系決めや,通信経路選択(ルーティング)の役割を持つ.

データリンク層は,物理層で直接接続されたノード間での通信を可能にする.
0と1の数字の列を意味のあるかたまり(フレーム)に分けて,相手に伝える.

物理層は,ビットの列(0と1の数字の列)を物理的な信号に変換し,実際の通信媒体を通じて送信する.

これらのそれぞれ独立した役割を持つ7つの階層が,互いに連携して通信を実現している.

\subsection{各層のデータ通信}
送信側の各階層でデータが処理される際,上位の階層から受け取ったデータに,その階層独自の情報(ヘッダ)を付加して下位の階層に引き渡す.
ヘッダには,その階層での通信制御やエラー検出,送信先情報など,通信を適切に管理するための情報が含まれている.
また,データリンク層ではデータの末尾にトレーラを付加する場合がある.
トレーラには,データが伝送中に損傷していないかを確認するためのエラー検出用の情報が含まれる.
一方で,受信側ではこれが逆順に処理され,各階層で対応するヘッダやトレーラが解析されることで,元のデータがアプリケーション層に復元される.





\section{評価}
\subsection{}



\clearpage



\section{謝辞}
本研究の遂行及び本論文の作成にあたり,須田研究室の仲間に多くの手助けを頂きました,深く感謝の意を表します.そして,本論文の作成にあたり多大なる御指導及び御助言を頂きました,須田宇宙准教授に深く感謝の意を表します.

\clearpage

%参考文献
\begin{thebibliography}{99}
\bibitem{TCP/IP} (1994) 竹下隆史・松山公保・荒井透・苅田幸雄: ``マスタリングTCP/IP 入門編第5版''
\bibitem{naikakufu2019} 内閣府: ``少子化対策の現状'', \url{https://www8.cao.go.jp/shoushi/shoushika/whitepaper/measures/w-2016/28webhonpen/html/b1_s1-1-3.html}, 2023/1/16参照
\bibitem{doukou}国立社会保障・人口問題研究所:``第15回出生動向基本調査'',
\url{https://www.ipss.go.jp/ps-doukou/j/doukou15/NFS15_report3.pdf},2023/1/14参照
\bibitem{prtimes}PRTIMES:``マッチングアプリは怖い?危ない目に合った?初めて会うまでの期間は!?徹底調査'',
\url{https://prtimes.jp/main/html/rd/p/000000016.000059676.html},2023/1/14参照
\bibitem{huuhuutyousa}「いい夫婦の日」に関するアンケート調査:``明治安田生命 NEWS RELEASE'',
\url{https://www.meijiyasuda.co.jp/profile/news/release/2022/pdf/20221116_01.pdf},2023/1/16参照
\bibitem{wecsocketsetumi}WebSocketとは?WebSocketについて詳しく解説します:``Web会議の基礎知識'',
\url{https://www.freshvoice.net/knowledge/word/6323/},2023/1/10参照
\end{thebibliography}

\end{document}
